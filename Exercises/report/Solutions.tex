\documentclass{article}
\usepackage[utf8]{inputenc}
\usepackage[a4paper, margin=3cm]{geometry}
\usepackage{amsmath}
\usepackage{mathtools}

\title{Computational Physics: Molecular Dynamics Simulations, Assignment 1}
\author{Vasco Ferreira}
\date{2025}

\begin{document}
\maketitle

\section*{1. Halley's comet}
\subsection*{a) Generalized velocity}
It would be prone to error to make simulations with would work with value to the order or $10^{-11}$, for the gravitational constant, and values of order $10^{+11}$, one astronomical unit. Therefore before anything is done, space will be rescaled by $1AU$ and time by one year ($1Y$), as follows.
\begin{equation*}
  \begin{cases}
    r \rightarrow &R=\frac{r}{1AU}\\
    t \rightarrow &T=\frac{t}{1Y}
  \end{cases}
\end{equation*}

With this, the gravitational acceleration formula can be rewritten

\begin{align*}
  \frac{d^2r}{dt^2} &= - \frac{GM}{r^3}\hat{r}\\
  \frac{d^2R}{dT^2} &= - \frac{(1Y)^2}{(1AU)^3}\frac{GM}{R^3}\hat{R}\\
  \frac{d^2R}{dT^2} &= - \Gamma\frac{GM}{R^3}\hat{R}
\end{align*}
With $\Gamma\approx39,39$. The position, velocity, and acceleration of the Verlet algorithm (eqs. (118), (119) from the lecture notes) can be generalized to 2D as follows
\begin{align*}
  &\begin{cases}
    \vec{x}=(x,y)\\
    \vec{v}=(v_x,v_y)\\
    \vec{a}=(a_x,a_y)\\
  \end{cases}\\
  &\begin{cases}
    a_x&= -\Gamma \frac{x}{\sqrt{x^2+y^2}^3}\\
    a_y&= -\Gamma \frac{y}{\sqrt{x^2+y^2}^3}
  \end{cases}\\
\end{align*}
These positions and velocities are all rescaled as indicated above, meaning that the initial conditions are as follows $\vec{x}(t=0)=(35.2,0)$ and $\vec{v}(t=0)=(0,0.1920952)$

\subsection*{b)}
Simulation implemented in jupyter notebook, $dt=0.01$ (equivalent to $3.6$ days). When trying $dt=0.1$ accumulated error was too big and comet wasn't in orbit for more than 1 period.

\subsection*{c)}


\section*{2. Symplectic vs.non-symplectic integrators}
\subsection*{a)}
For the Euler integrator, it can be directly seen from equation (123) from the lecture notes that the Jacobian of the time transformation is 
\begin{equation*}
  M = 
  \begin{pmatrix}
    1 & \Delta t\\
    -\Delta t & 1
  \end{pmatrix}
\end{equation*}
From this we can easily see that $\det(M)=1+\Delta t^2>1$ for $\Delta t>0$.\\
The same can be done for the sympletic integrator from equation (124) of the lecture notes, but an extra step needs to be taken. After rewriting the expression for $q(t+\Delta t)$ as
\begin{equation*}
  q(t+\Delta t) = q(t) + (p(t)-q(t)\Delta t)\Delta t
\end{equation*}
We can find that the Jacobian of the time transformation for this integrator is.
\begin{equation*}
  M_s = 
  \begin{pmatrix}
    1-\Delta t^2 & \Delta t\\
    -\Delta t & 1
  \end{pmatrix}
\end{equation*}
For the sympletic integrator we have that $\det(M_s)=1-\Delta t^2+\Delta t^2 = 1$.

\subsection*{b)} To show that $H'$ is a constant of motion I'll show that $H'(t+\Delta t)=H'(t)$. 
\begin{align*}
  H'(t+\Delta t) & = \frac{(p(t+\Delta t)^2+ q(t+\Delta t)^2)}{2} -\frac{p(t+\Delta t)q(t+\Delta t)}{2}\Delta t\\
  p(t+\Delta t)^2 &= p(t)^2 -2p(t)q(t)\Delta t + q(t)^2\Delta t^2\\
  q(t+\Delta t)^2 &= p(t)^2\Delta t^2 +2p(t)q(t)(\Delta t - \Delta t^3)+q(t)^2(1-\Delta t^2)\\
  p(t+\Delta t)^2+ q(t+\Delta t)^2&= p(t)^2 (1+\Delta t^2)-2p(t)q(t)\Delta t^3+q(t)^2(1-\Delta t^2+\Delta t ^4)\\ 
  p(t+\Delta t)q(t+\Delta t) &= p(t)^2\Delta t +p(t)q(t)(1-2\Delta t^2)-q(t)^2(\Delta t - \Delta t^3)\\
                             &\Rightarrow\\ 
  2H'(t+\Delta t)&=p(t)^2-p(t)q(t)\Delta t + q(t)^2\\ 
                 &= 2H'(t)
\end{align*}
Therefore it holds that $H'=H-\frac{pq}{2}\Delta t$ is a constant of motion.
\subsection*{c)} SIMULATION $\Rightarrow$ TODO

\section*{3.}
    


\end{document}
